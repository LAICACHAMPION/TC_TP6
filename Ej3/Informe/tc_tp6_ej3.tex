\documentclass[../../tc_tp6_main.tex]{subfiles}

\begin{document}

%capítulo
\chapter{Dise\~no de VCO}

En esta secci\'on, se propondr\'a una implementaci\'on de un oscilador controlado por tensi\'on, mejor conocido como VCO (por sus siglas en ingl\'es: \textit{voltage-controlled oscillator}). El mismo recibir\'a como entrada una tensi\'on constante de entre 0V y 5V, y producir\'a a la salida una se\~nal senoidal de 1V de amplitud, cuya frecuencia variar\'a linealmente seg\'un la tensi\'on de entrada entre 1kHz y 10kHz. 


\section{Introducci\'on: el VCO}

\section{Dise\~no}

\begin{figure}[H]
	\centering
	\begin{circuitikz}
		\draw
		(5,7) node[op amp, yscale=-1] (oa) {}		
		(10,6.5) node[op amp] (comp) {}			
		(3,1.5) node[npn, xscale=-1](npn){}		
		
		(0,7.5) node[left] {$V_{IN}$}
		(0,7.5) to [short, o-*] (1,7.5) 
		to [R = 10k$\Omega$, -*] (3, 7.5)
		to [short] (oa.+)	
		(3, 7.5) to [R, l_=10k$\Omega$] (3, 9.5) node[ground, yscale=-1]{}	
		
		(1, 7.5) to [short] (1, 6.5)
		to [R, l_=$2R$, -*] (3, 6.5)
		to [short] (oa.-)
		(3, 6.5) to [short, -*] (3, 5.5)
		to [R = $R$] (npn.C)
		
		(comp.-) to [short, -*] (oa.out)
		to [short] (oa.out) -- ++ (0, -1.5)
		to [C=$C$] (3, 5.5)
		
		(oa.out) to [short, -o] (6.2, 7.5) node[above]{$V_{TR}$}	
		
		(comp.+) to [short] (8, 6)
		to [short, -*] (8,5)
		to [R, l_=1k$\Omega$] (8,3) node[ground]{}
		
		(8,5) to [R, l_=2.2k$\Omega$, -*] (11.5,5)
		to [short, -*] (11.5, 6.5) to [short] (comp.out)
		(11.5, 6.5) to [short, -o] (12, 6.5) node[right]{$V_{SQ}$}
		
		(11.5, 5) to [short] (11.5, 1.5)
		to [R=1k$\Omega$] (npn.B)
		
		(npn.E) node[ground]{}		
	;\end{circuitikz}
	
	\caption{\textit{Voltage-controlled oscillator} con salidas triangular y cuadrada}
\end{figure}

%% aca pongo el VCO posta. OH POR DIOS NECESITO MAS ETAPAS

\subsection{Conversor triangular a senoidal}

\begin{figure}[H]
	\centering
	\begin{circuitikz}
		\draw
		(5,5) node[npn](n1){}		
		(8,5) node[npn, xscale=-1](n2){}		

		(1.5, 5) node[left]{$V_{TR}$}
		to [vR=100k$\Omega$, o-*] (4, 5) 
		to [R, l_=2.2k$\Omega$] (4, 3) node[ground]{}
		(4,5) to [short] (n1.B)		
		
		(n1.E) to [short, -*] (5, 3.5)
		to [R=390R, -*] (8,3.5)
		to (n2.E)
		
		(5, 3.5) to [short] (5, 1.5)
		to [R=100k$\Omega$] (8,1.5)
		to [short] (8, 3.5)
		
		(6.5, -1) node[below]{$V_{EE}$}
		to [R=18k$\Omega$, o-] (6.5, 1.3) node[inputarrow, rotate=90]{}
		
 		(n2.B) to [short, -*] (9,5)
 		to [R=2.2k$\Omega$] (9,3) node[ground]{}
 		
 		(n1.C) to [R=10k$\Omega$] (5,8.5)
 		to [short] (8,8.5)
 		to [R=10k$\Omega$] (n2.C)
 		
 		(6.5,8.5) to [short, *-o] (6.5, 9) node[above]{$V_{CC}$}
 		
 		(11,6) node [op amp, scale=0.5](oa) {}
 		(oa.+) to [short, -*] (8, 5.75)
 		(oa.-) to [short, -*] (5, 6.25) 		
 		(oa.+) to [short] (10, 5.75) node[ground]{}
 		
		(oa.-) to [short] (10, 6.25)
		to [short, *-] (10, 7)
		to [vR=5k$\Omega$] (12,7)
		to [short, -*] (12, 6) 
		to [short, -o] (12.5,6) node [right]{$V_{OUT}$} 		
 		(oa.out) to [short] (12,6)
 		
	;\end{circuitikz}
	
	\caption{Conversor de onda triangular a senoidal}
\end{figure}

\section{Resultados}

\section{Conclusiones}

DIO TODO IMPECABLE VIEJA NI NO VIMO


\section{Anexo}
\subsection{Transformaci\'on lineal [0V;5V] a [1V;10V]}


\begin{figure}[H]
	\centering
	\begin{circuitikz}
		\draw
		(4,3) node[op amp, yscale=-1](oa){}		
		
		(0,5) node[left]{+15V} 
		to [R, l^=$R_1$, o-] (2.8,5)
		to [short] (oa.+)
		
		(0,3.5) node[left]{$V_{IN}$}
		to [R, l^=$R_2$, o-*] (oa.+)
		
		(oa.-) to [short, -*] (2.8,1.5)
		to [R=$R_4$] (2.8, -0.5) node[ground]{}
		
		(2.8, 1.5) to [R=$R_3$] (5.2, 1.5)
		to [short, -*] (oa.out) 
		to [short, -o] (5.7, 3) node[right] {$V_{OUT}$} 
	;\end{circuitikz}
	\caption{Sumador no inversor}
\end{figure}
\end{document}
